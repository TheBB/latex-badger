\documentclass{beamer}

\usetheme{CambridgeUS}
\usecolortheme{orchid}

\usepackage{alltt}
\usepackage{xcolor}
\usepackage{wrapfig}
\usepackage{siunitx}
\usepackage{inconsolata}
\usepackage{amsmath}
\usepackage{bm}
\usepackage{nicefrac}
\usepackage[font+=small]{caption}
\usepackage{tikz}
\usetikzlibrary{shapes.geometric, positioning}
\usepackage{pgfplots}

\begin{document}

\title[Badger]{Badger: automated and repeatable test cases}
\author[E. Fonn]{
  E.~Fonn\inst{1,2} \and
  A.~M.~Kvarving\inst{2} \and
  T.~Kvamsdal\inst{3} \and
}
\institute[SINTEF]{
  \inst{1}%
  \url{eivind.fonn@sintef.no}
  \and \inst{2}%
  Applied Mathematics, SINTEF ICT
  \and \inst{3}%
  Department of Mathematical Sciences, NTNU
}
\date[October 2015]{}

\newcommand{\rmc}{{\text{c}}}
\newcommand{\rmf}{{\text{f}}}
\newcommand{\rmt}{{\text{t}}}
% \newcommand{\rms}{{\text{s}}}
\newcommand{\rmu}{{\text{u}}}
\newcommand{\rmq}{{\text{q}}}
% \newcommand{\rmp}{{\text{p}}}
\newcommand{\rmv}{{\text{v}}}

\titlegraphic{\includegraphics[width=0.3\textwidth]{common/sintef}}

\begin{frame}
  \titlepage{}
\end{frame}

\section{The problem}

\begin{frame}
  \frametitle{The problem}

  \begin{itemize}
  \item Test cases are built, run, published and then forgotten about
  \item Reproducing a case from memory is nigh on impossible
  \item If scripts do exist, they are procedural and not declarative
  \end{itemize}
\end{frame}

\section{Badger}

\begin{frame}
  \frametitle{Badger}

  Badger is a BATCH job runnER, specialized for numerical applications.

  \begin{itemize}
  \item Test cases are specified in a simple declarative language
  \item A powerful template language allows us to reuse the same setup files for
    many different instances.
  \item Ties in easily with regression testing and continuous integration tools.
  \end{itemize}
\end{frame}

\section{Example}

\begin{frame}[fragile]
\begin{verbatim}
executable: /home/eivindf/.../PoroElasticity
cmdargs: -2D -mixed terzhagi.xinp
templates: terzhagi.xinp
variables:
  degree: [1, 2, 3]
  elements: [8, 16, 32]
constants:
  endtime: 10
parse:
  - !!str 'Relative \|p-p^h\|_l2 : (?P<p_rel_l2>[^\s]+)'
  - !!str 'Total time\s+\|\s+(?P<cpu_time>[^\s]+)
           \s+\|\s+(?P<wall_time>[^\s]+)'
types:
  p_rel_l2: float
  cpu_time: float
  wall_time: float
\end{verbatim}
\end{frame}

\begin{frame}[fragile]
  The template file contains lines such as the following (for IFEM).

\begin{verbatim}
<raiseorder patch="1" u="{{ degree-1 }}" v="{{ degree-1 }}" />

<refine type="uniform" patch="1"
        u="{{ (elements//8)-1 }}" v="{{ elements-1 }}" />

<timestepping start="0" end="{{ endtime }}"
              dt="{{ 1.0/elements**(degree+1) }}" />
\end{verbatim}

  The underlying language is arbitrary, here XML as an example.
\end{frame}

\section{Feature wishlist}

\begin{frame}
  \frametitle{Feature wishlist}
  
  \begin{itemize}
  \item Running pre- and post steps. (Geometry generation, plotting, etc.)
  \item Running a sparse subset of all possible test cases.
  \item Running several cases in parallel with near optimal distribution.
  \item Reading data from generated files and not just standard out.
  \item An easier way to read output than having to write regular expressions,
    which are difficult and error-prone.
  \item Options to execute dry runs and verbose runs.
  \item Other output formats.
  \item Grabbing multiple data points from each run instead of just the last
    match for each regular expression.
  \item Incremental output.
  \end{itemize}
\end{frame}

\end{document}
